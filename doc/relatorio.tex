\documentclass{article}
\usepackage[T1]{fontenc}
\usepackage[utf8]{inputenc}
\usepackage[brazil]{babel}

\usepackage{amsmath}
\usepackage{listings}
\usepackage{tikz}
\usetikzlibrary{positioning}
\usetikzlibrary{shapes}

\newcommand{\ask}{\lstinline"ASK_QUESTION"}
\newcommand{\guess}{\lstinline"GUESS_ANSWER"}
\newcommand{\restart}{\lstinline"RESTART"}
\newcommand{\giveup}{\lstinline"GIVE_UP"}
\newcommand{\newquestion}{\lstinline"ASK_NEW_QUESTION"}
\newcommand{\unguesser}{\lstinline"Unguesser"}

\lstset{
    basicstyle=\ttfamily
}

\begin{document}

\title{
    INE5430 --- Inteligência Artificial \\
    Unguesser: jogo de adivinhações
}
\author{
    \begin{tabular}{r l}
        Lucas Berri Cristofolini & 12100757 \\
        Tiago Royer & 12100776 \\
        Wagner Fernando Gascho & 12100779
    \end{tabular}
}

\maketitle

\section{Máquina de Estados}

A classe principal do programa, \unguesser,
controla a ordem das operações.
É esta classe que escolha qual a próxima pergunta a ser feita,
quando e quem será chutado,
e quando desistir ou pedir ajuda.

Esta classe é melhor entendida como o autômato finito
descrito da figura \ref{unguesser-automata}.

\begin{figure}[h]
    \centering
    \begin{tikzpicture}[-stealth]
        \begin{scope}[every node/.style={shape=ellipse,draw}]
            \node (ask) {\ask};
            \node (guess) [below = of ask] {\guess};
            \node (giveup) [right = of guess] {\giveup};
            \node (newquestion) [below = of giveup] {\newquestion};
            \node (restart) [below = of guess] {\restart};
        \end{scope}

        \begin{scope}[auto]
            \draw (ask) to[loop above] node {1} (ask);
            \draw (ask) to[bend left] node{2} (guess);
            \draw (guess) to[bend left] node{3} (ask);
            \draw (guess) to node{4} (giveup);
            \draw (guess) to node{5} (restart);
            \draw (giveup) to node{6} (newquestion);
            \draw (giveup) to node{7} (restart);
            \draw (newquestion) to node{8} (restart);
            \draw (node cs:name=restart, angle=180) to [bend left = 45] node{9}
                  (node cs:name=ask, angle=185);
        \end{scope}
    \end{tikzpicture}

    \caption{Máquina de estados que corresponde à classe \unguesser.}
    \label{unguesser-automata}
\end{figure}

A máquina de estados começa no estado \ask.
Neste estado, o programa decide fazer uma nova pergunta ao usuário.
(O algoritmo de escolha da próxima pergunta está descrito na seção \ref{novapergunta}.)
Após receber a resposta, o sistema ranqueia todas as entidades no banco de dados
de acordo com o algoritmo descrito na seção \ref{ranqueamento}.
O programa também atualiza o limiar de aceitação;
este valor decide quais as entidades serão consideradas ``válidas''
(isto é, com boas chances de ser a resposta correta).

O limiar de aceitação começa bastante baixo e vai tornando-se mais estrito
conforme passa o jogo.
Enquanto não há uma única resposta,
a máquina de estados mantém-se no estado \ask,
através da transição 1.
Caso o limiar de aceitação seja estrito o bastante
para apenas uma entidade ser aceitável,
o programa executa a transição 2,
e passa ao estado \guess.
(O programa também faz esta transição
quando não não há mais questões no banco de dados;
é uma ``última tentativa'' de acertar o que o usuário estava pensando.)

No estado \guess, o programa pega a entidade melhor ranqueada
e mostra ao usuário.
Caso o chute esteja certo, o programa faz a transição 5.
Ele atualiza o banco de dados,
incorporando as respostas que o usuário acabou de dar,
e transita para o estado \restart.

Caso contrário (nos chutamos errado),
o usuário pode escolher entre nos dar uma nova chance
(transição 3, voltamos ao \ask)
ou nos dizer a resposta
(transição 4, partimos para o \giveup).
Nós também tomamos a transição 4 se não há mais perguntas no banco de dados.

Antes de voltarmos para \ask, nós reduzimos o limiar
para considerar uma entidade aceitável.
Dessa forma, ficaremos um tempo fazendo novas perguntas
antes de tentar outra entidade.

No estado \giveup, nós desistimos de tentar acertar a entidade do usuário,
e pedimos a ele que nos diga no quê ele estava pensando.
O usuário então digita o nome da entidade.
Caso seja uma entidade que ele conhece, o programa atualiza o banco de dados.

Normalmente, o programa tomaria a transição 7 agora,
indo parar no estado \restart.
Entretanto, pode ser que o banco de questões esteja muito ``frágil''.
Caso cheguemos neste estado sem reduzir a quantidade de entidades válidas para 1,
faremos a transição por 6, até o estado \newquestion.

Em \newquestion, nós iremos mostrar uma lista de entidades
(as melhores ranqueadas do banco de dados)
e pedir que o usuário nos ajude a distingui-las.
O programa pede que usuário digite uma questão que as distinga,
e as respostas à esta pergunta para cada uma das entidades.
Então, adicionamos esta questão e as outras respostas no banco de dados.
Transitamos por 8 e chegamos em \restart.

Em \restart, a única coisa que podemos fazer é recomeçar o jogo
(transição 9)
com outra entidade.
\footnote{
    As interfaces com o usuário que implementamos
    não suportam este recomeço;
    elas jogam apenas um jogo por execução.
}

\section{Ranqueamento \label{ranqueamento}}

Nosso algoritmo de ranqueamento é baseado em álgebra linear.
O produto interno entre dois vetores está relacionado com o conceito de ortogonalidade.
Dois vetores não relacionados possuem produto interno nulo;
dois vetores positivamente relacionados possuem produto interno positivo
e dois vetores negativamente relacionados possuem produto interno negativo.

Consideramos cada entidade no banco de dados como um vetor.
Cada elemento do vetor corresponde à resposta que esta entidade dá à disciplina;
$1$ corresponde ao ``sim'', $-1$ corresponde ao ``não''.
Valores entre $-1$ e $1$ representam a noção de incerteza;
por exemplo, nós codificamos ``talvez'' com $0.5$,
``improvável'' com $-0.5$, e ``não sei'' com $0$.

A lista de respostas que o usuário já deu
(o vetor parcial das respostas)
também está neste mesmo formato.
Ao fazer o produto interno deste vetor com uma entidade no banco de dados,
nós calculamos o quão relacionada a entidade está com o vetor de respostas.
Esta é a pontuação da entidade.

O programa ordena as entidades através desta pontuação.
Quanto maior a pontuação, maior é a chance de o vetor de respostas parcial
ser aquela entidade.

O limiar de aceitação é uma média ponderada entre a maior pontuação
e a menor pontuação.
Todas as entidades com pontuação acima do limiar são consideradas
``aceitáveis''.

O cálculo da média usa a seguinte fórmula:
\begin{equation}
    \text{Limiar de aceitação} =
    (1-t) * (\text{maior valor}) + t * (\text{menor valor}).
\end{equation}
Quanto mais próximo de $0$ é $t$, maior é o limiar de aceitação.
$t$ inicia o programa em $0.5$.
Conforme o programa avança,
$t$ fica cada vez menor,
até o limiar de aceitação ser suficiente para tornar apenas uma entidade aceitável.
Então, chutamos enta entidade.
Caso tenhamos errado, aumentamos $t$ para reduzir o limiar,
e começamos de novo.

\subsection{Escolha de novas perguntas}

O algoritmo de escolha de novas perguntas foi construido para tentar,
simultaneamente,
criar distinções entre as entidades mais ``aceitáveis'',
mas, ao mesmo tempo, lidar com a possibilidade de
a entidade certa ter sido erroneamente classificada como ``não-aceitável''
(talvez pelo fato de as perguntas anteriores terem sido ruins
ou por que o usuário errou alguma das questões).

Para garantir a segunda exigência, o algoritmo acabou sendo não-determinístico.
Para tentar a primeira, o algoritmo usa um gerador ponderado.

Primeiro, o algoritmo constrói a ``questão ideal''.
Para cada questão, ele soma todas as respostas positivas e todas as negativas,
separando as positivas das negativas.
Quanto maior for a diferença entre estes dois valores, melhor é a questão
(isto é, maior sua seletividade).
Entretanto, o algoritmo considera apenas as entidades que estão acima
do limiar de aceitação para somar as respostas.

O algoritmo soma o máximo das respostas positivas e o máximo das respostas negativas
e divide por dois.
Cada questão, então, é comparada, usando distância euclidiana,
contra esta ``questão ideal''.
Quanto mais desequilibrada e menos seletiva for a questão,
menor é a chance de ser escolhida.

Desta forma, o algoritmo tem alta probabilidade de escolher uma questão
que irá discernir a entidade certa no banco de dados,
mas sem negligenciar a possibilidade de o usuário ter errado alguma resposta.

\end{document}
